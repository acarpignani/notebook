\documentclass[x11names, tikz]{standalone}

\begin{document}
\begin{tikzpicture}
% First layer
\fill[fill=Orange1, opacity = 0.7] 
	(0,0) -- ++(4,0) -- ++(150:1.75) -- ++(-4,0) -- cycle;
\node[Orange1, left] at (150:2) {Data};
% Second layer
\begin{scope}[yshift = 0.5cm]
\fill[fill=SeaGreen4, opacity = 0.7] 
	(0,0) -- ++(4,0) -- ++(150:1.75) -- ++(-4,0) -- cycle;
\node[SeaGreen4, left] at (150:2) {Aesthetics};
\end{scope}
% Third layer
\begin{scope}[yshift = 1.0cm]
\fill[fill=SteelBlue3, opacity = 0.7] 
	(0,0) -- ++(4,0) -- ++(150:1.75) -- ++(-4,0) -- cycle;
\node[SteelBlue3, left] at (150:2) {Geometries};
\end{scope}
% Fourth layer
\begin{scope}[yshift = 1.5cm]
\fill[fill=Purple4, opacity = 0.7] 
	(0,0) -- ++(4,0) -- ++(150:1.75) -- ++(-4,0) -- cycle;
\node[Purple4, left] at (150:2) {Statistics};
\end{scope}
% Fifth layer
\begin{scope}[yshift = 2.0cm]
\fill[fill=Firebrick3, opacity = 0.7] 
	(0,0) -- ++(4,0) -- ++(150:1.75) -- ++(-4,0) -- cycle;
\node[Firebrick3, left] at (150:2) {Facets};
\end{scope}
% Sixth layer
\begin{scope}[yshift = 2.5cm]
\fill[fill=Burlywood3, opacity = 0.7] 
	(0,0) -- ++(4,0) -- ++(150:1.75) -- ++(-4,0) -- cycle;
\node[Burlywood3, left] at (150:2) {Coordinates};
\end{scope}
% Seventh layer
\begin{scope}[yshift = 3.0cm]
\fill[fill=Gold1, opacity = 0.7] 
	(0,0) -- ++(4,0) -- ++(150:1.75) -- ++(-4,0) -- cycle;
\node[Gold1, left] at (150:2) {Themes};
\end{scope}
\end{tikzpicture}
\end{document}